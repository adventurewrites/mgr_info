\chapter{Porównanie testowalności na przykładzie aplikacji}
\label{analiza_testow}

\section{Opis doświadczenia}
Na przykładzie jednej z aplikacji postaram się przedstawić różnicę w podejściu do testowania w przypadku programu dla systemu Android napisanego w architekturze „standardowej” i tego samego programu napisanego przy użyciu Clean Architecture.

Do tego celu autor wykorzysta aplikację \textit{JSON Web Token Authentication for Android} napisaną przez Victora Albertosa , a której źródła udostępnione są w serwisie GitHub na licencji \textit{open source}.

https://github.com/VictorAlbertos/RestAPIParseAuthAndroid

https://github.com/VictorAlbertos/RestAPIParseAuthCleanAndroid

http://victoralbertos.com/authentication-for-android-and-ios-tutorial-1-creating-parse-project-and-checking-endpoints/



\section{Aplikacja napisana standardowo}

\section{Aplikacja napisana z wykorzystaniem \textit{clean architecture}}

\section{Wyniki doświadczenia}
