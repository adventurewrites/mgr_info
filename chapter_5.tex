\chapter{Porównanie testowalności na przykładzie aplikacji}
\label{analiza_testow}

\section{Opis doświadczenia}
Na przykładzie jednej z aplikacji postaram się przedstawić różnicę w podejściu do testowania w przypadku programu dla systemu Android napisanego w architekturze „standardowej” i tego samego programu napisanego przy użyciu Clean Architecture.

Do tego celu autor wykorzysta aplikację \textit{JSON Web Token Authentication for Android} napisaną przez Victora Albertosa , a której źródła udostępnione są w serwisie GitHub na licencji \textit{open source}.

https://github.com/VictorAlbertos/RestAPIParseAuthAndroid

https://github.com/VictorAlbertos/RestAPIParseAuthCleanAndroid

http://victoralbertos.com/authentication-for-android-and-ios-tutorial-1-creating-parse-project-and-checking-endpoints/



\section{Zastosowanie \textit{The Clean Architecture} przy tworzeniu aplikacji androidowych}
\label{clean_android}
Każde z wymienionych w poprzednim rozdziale podejść do uporządkowania architektury systemowej można wykorzystać do polepszenia testowalności aplikacji tworzonych dla tego systemu. Analizując je krok po kroku można dojść do wniosku, że ich idea jest taka sama, a różnią się szczegółami. Dla celów tej pracy wystarczy wybrać jedną z nich - w tym przypadku autor zdecydował się na \textit{The Clean Architecture}.

Poniżej przedstawiono charakterystykę każdej z warstw z uwzględnieniem szczegółów, nad którymi należy się szczególnie pochylić projektując aplikację dla systemu Android.

\section{Testowalność warstwy logicznej obu aplikacji}
\label{android_entities}
Analizując warstwę kluczową aplikacji Victora Albertosa można wyodrębnić następujące kluczowe elementy, które nie powinny ulec zmianie:
\begin{itemize}
\item
aaa
\end{itemize}

\section{Warstwa użytkowa - \textit{Use Cases}}
\label{android_use_cases}
\subsection{Odwołanie do bazy danych}

\section{Warstwa \textit{Persistence - Data}}

\section{Warstwa UI}
\subsection{Pattern Model-View-Presenter}

\section{Spinanie do kupy - \textit{Dependency Injection}}

\section{Wyniki doświadczenia}
\label{wyniki_doswiadczenia}