\chapter*{Abstrakt}
\label{abstract}

\textbf{TESTABILITY OF MOBILE APPLICATIONS FOR ANDROID}

\vspace{20pt}

\textbf{Author:} Rafał Sowiak

\vspace{\stretch{2}}
\textbf{Faculty:} WEEIA

\vspace{\stretch{2}}
\textbf{Institute:} DMCS

\vspace{\stretch{2}}
\textbf{Tags and key words:} testability, Android, Clean Architecture, Test Driven Development, TDD

\vspace{\stretch{2}}
\textbf{Abstract:} 
\textit{The term "android" is mainly associated with the market of modern electronic devices. These include smartphones, tablets, netbooks, GPS receivers, watches, as well as televisions, refrigerators, washing machines and other equipment used now. As the operating system available for free, Android connects a huge community of developers writing applications which extend the functionality of these devices. Provides to developers a rich user environment, including access to the Internet, database servers, printers and other peripherals. Getting rid of the possibility of the use of unit testing at an early stage of the project because of poorly designed structure of the application, designers risk losing quality. And as a consequence - risk losing of customer confidence in the software. The aim of this study is to present an alternative approach to software architecture intended for system Android, as well as to present an alternative method of approach to the whole process of writing software, which would help in increasing the testability of software on the system.}

\vspace{\stretch{2}}
The observations of the author shows that it is very easy to write bad and untestable code for Android Operational System. However, project managers and programmers should know, that using of certain processes and methodologies may significantly improve testability of applications for Android.

In the first part of the publication autor presents the current, widely used approach to application development for Android. The first chapter contains general information about Android: history of emergence and development of the system, analysis of the current popularity in the segment of portable devices, as well as a description of the possibilities it offers to developers. The second chapter focuses on testing. Will be reminded of the basic test concepts, the definition of testability and maintainability of software, conducted the analysis of the test areas and types of tests and the introduction of agile approaches in the verification process. In the third chapter the author describes the key issue of this publication, which is the problem of testability of applications written for Android. Reminds concepts of software and system architecture and describes the architecture currently used in Android applications. Then he exposes the difficulties encountered during testing of this type of architecture.

Part two is a proposal how to solve the problem outlined in chapter three. In the fourth chapter the author proposes the use of technique \textit{Test Driven Development} in the development process and encourages the use of one of the three described concepts of the system architecture: \textit{The Clean Architecture} proposed initially by Robert Cecil Martin. Proposes also Agile approach when modifying existing applications. The fifth chapter is the analysis of testability based one of the ready-made applications, once written in standard architecture, and twice - using the \textit{The Clean Architecture} and \textit{TDD} approach.


