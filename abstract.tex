\chapter*{Abstrakt}
\label{abstract}

\begin{center}
\textbf{TESTOWALNOŚĆ APLIKACJI MOBILNYCH NA PLATFORMĘ ANDROID}

\textbf{STRESZCZENIE}
\end{center}

Jako system operacyjny dostępny bezpłatnie, Android zrzesza ogromną społeczność programistów piszących aplikacje, które poszerzają funkcjonalność urządzeń mobilnych. Okazuje się jednak, że temat automatycznego testowania opracowywanych aplikacji jest często pomijamy w~literaturze, a~stosowane powszechnie podejście do tworzenia nowych aplikacji nie pozwala na sprawne pisanie testów jednostkowych. W~konsekwencji programiści rozpoczynają testowanie od~testów integracyjnych. Wychodząc tym problemom naprzeciw, autor w~tej pracy dowodzi, że zastosowanie odpowiedniej architektury oraz odpowiedniego podejścia do pisania oprogramowania może ułatwić testowanie aplikacji napisanych dla tego systemu.  

Poniższa praca zawiera analizę testowalności aplikacji na platformę Android. Wprowadzone i~wyjaśnione zostają podstawowe pojęcia związane z~programowaniem na tę platformę oraz testowalnością oprogramowania, a następnie opisany proces projektowy, który znacząco ułatwia utrzymanie i~opracowywanie aplikacji Android. Porównane zostają również wybrane rozwiązania architektoniczne w~kontekście automatycznego testowania. Otrzymane wyniki pozwalają na wyciągnięcie ciekawych wniosków.

\textbf{Słowa kluczowe:} testowalność, Android, Clean Architecture, Test Driven Development

\begin{center}
\textbf{TESTABILITY OF MOBILE APPLICATIONS FOR ANDROID}

\textbf{ABSTRACT}
\end{center}

As the operating system available for free, Android joins a~huge community of developers writing applications that extend the functionality of mobile devices. However, it turns out that subject of automatic testing of Android applications is often omitted in the literature, and commonly used approach to develop new applications doesn't help for unit tests writing. As a~result, developers often use the integration test instead. Going into the problem, the author demonstrates that using of appropriate architecture and approach to writing software can facilitate the testing of applications written for this system.

This publication presents an analysis of applications testability on Android platform. The author introduces and explains the basic concepts related to programming on the platform, as well as software testability. In the second part, the design process, which significantly facilitates the maintenance and development of Android applications, is described. Also selected architectural solutions are compared in the context of automated testing. The results allow to draw many interesting conclusions.

\textbf{Key words:} testability, Android, Clean Architecture, Test Driven Development


