\chapter*{Abstrakt}
\label{abstract}

\begin{center}
\textbf{TESTOWALNOŚĆ APLIKACJI MOBILNYCH NA PLATFORMĘ ANDROID}

\textbf{STRESZCZENIE}
\end{center}

Jako system operacyjny dostępny bezpłatnie, Android zrzesza ogromną społeczność programistów piszących aplikacje, które poszerzają funkcjonalność urządzeń mobilnych. Dostarcza programistom bogate środowisko użytkownika pozwalając pisać oprogramowanie bez konieczności sięgania do niższych warstw systemu, włączając możliwość dostępu do Internetu, serwerów baz danych, drukarek i~innych urządzeń peryferyjnych. Okazuje się jednak, że temat automatycznego testowania opracowywanych aplikacji jest często pomijamy w~literaturze, a~stosowane powszechnie podejście do tworzenia nowych aplikacji nie pozwala na sprawne pisanie testów jednostkowych. W~konsekwencji programiści często korzystają z~testów integracyjnych, gdzie nakład pracy jest dużo większy. Wychodząc tym problemom naprzeciw, autor w~tej pracy dowodzi, że zastosowanie odpowiedniej architektury oraz odpowiedniego podejścia do pisania oprogramowania może ułatwić testowanie aplikacji napisanych dla tego systemu.  

Poniższa praca zawiera analizę testowalności aplikacji na platformę Android. W~pierwszej części autor wprowadza i~wyjaśnia podstawowe pojęcia związane z~programowaniem na tę platformę oraz testowalnością oprogramowania. W~części drugiej szczegółowo opisuje proces projektowy, który znacząco ułatwia utrzymanie i~opracowywanie aplikacji Android i~porównuje wybrane rozwiązania architektoniczne w~kontekście automatycznego testowania. Otrzymane wyniki są bardzo obiecujące i~pozwalają na wyciągnięcie wielu ciekawych wniosków.

\textbf{Słowa kluczowe:} testowalność, Android, The Clean Architecture, Test Driven Development, TDD, testy jednostkowe

\begin{center}
\textbf{TESTABILITY OF MOBILE APPLICATIONS FOR ANDROID}

\textbf{ABSTRACT}
\end{center}

As the operating system available for free, Android joins a~huge community of developers writing applications that extend the functionality of mobile devices. Android provides for them a~rich user environment, allowing to write software without having to reach into the lower layers of the system, including the ability to access the Internet, database servers, printers and other peripherals. However, it turns out that subject of automatic testing of Android applications is often omitted in the literature, and commonly used approach to develop new applications doesn't help for unit tests writing. As a~result, developers often use the integration test instead, where the workload is much greater. Going into the problem, the author demonstrates that using of appropriate architecture and approach to writing software can facilitate the testing of applications written for this system.

This publication presents an analysis of applications testability on Android platform. In the first part the author introduces and explains the basic concepts related to programming on the platform, as well as software testability. In the second part, describes in detail the design process, which significantly facilitates the maintenance and development of Android applications. Compares also the selected architectural solutions in the context of automated testing. The results are very promising and allow to draw many interesting conclusions.

\textbf{Tags and key words:} testability, Android, The Clean Architecture, Test Driven Development, TDD, unit tests


