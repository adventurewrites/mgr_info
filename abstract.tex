%\chapter*{Abstrakt}
%\label{abstract}

\begin{center}
\Large{\textbf{Streszczenie}}
\end{center}

\textbf{Tytuł:} Testowalność aplikacji mobilnych na platformę Android
\vspace{7 pt}

Jako system operacyjny dostępny bezpłatnie, Android zrzesza ogromną społeczność programistów piszących aplikacje, które poszerzają funkcjonalność urządzeń mobilnych. Okazuje się jednak, że temat automatycznego testowania opracowywanych aplikacji jest często pomijany w~literaturze, a~stosowane powszechnie podejście do tworzenia nowych aplikacji nie pozwala na sprawne pisanie testów jednostkowych. W~konsekwencji programiści rozpoczynają weryfikację od~testów integracyjnych. Wychodząc tym problemom naprzeciw, autor w~tej pracy dowodzi, że zastosowanie odpowiedniej architektury oraz odpowiedniego podejścia do pisania oprogramowania może ułatwić automatyczne testowanie aplikacji napisanych dla tego systemu.  

Wprowadzone zostają podstawowe pojęcia związane z~programowaniem na platformę Android oraz testowalnością oprogramowania. Następnie opisany zostaje proces projektowy, który znacząco ułatwia utrzymanie i~opracowywanie aplikacji Android. Porównane zostają również wybrane rozwiązania architektoniczne w~kontekście automatycznego testowania. Na podstawie przykładowej aplikacji opracowanej w dwóch różnych architekturach przeprowadzona jest analiza ilości niezbędnych testów oraz czasu ich wykonania. Otrzymane wyniki pozwalają na wyciągnięcie wniosków w kontekście polepszenia testowalności oprogramowania na platformę Android.
\vspace{7 pt}

\textbf{Słowa kluczowe:} testowalność, Android, Clean Architecture, Test Driven Development

\begin{center}
\Large{\textbf{Abstract}}
\end{center}

\textbf{Title:} Testability of Mobile Applications for Android
\vspace{7 pt}

As a free operating system, Android joins a~huge community of developers writing applications that extend the~functionality of~mobile devices. However, it turns out that subject of automatic testing of Android applications is often omitted in the literature. Commonly used approach to development of new applications does not allow for the efficient creation of unit tests. As a~result, developers often start a~verification process from integration tests. In this work author demonstrates that using of the appropriate architecture can facilitate the testing of applications written for this system.

The author introduces the basic concepts related to programming on the Android platform, as well as to software testability. He describes precisely the design process, which significantly facilitates the development and maintenance of Android applications, and compares selected architectural solutions in the context of automated testing. On the basis of the selected application in standard and modified version using \textit{The Clean Architecture} and TDD technique -  analysis of the number of necessary tests and the time of their execution is caried out. Obtained results are encouraging and useful for development of similar solutions.
\vspace{7 pt}

\textbf{Key words:} testability, Android, Clean Architecture, Test Driven Development


