\chapter*{Wstęp}
\label{wstep}

Pojęcie \textit{Android} kojarzone jest przede wszystkim z~rynkiem nowoczesnych urządzeń elektronicznych. Są to smartfony, tablety, netbooki, odbiorniki GPS, zegarki, a~także telewizory, lodówki, pralki i~inne urządzenia wykorzystywane obecnie. Jako system operacyjny dostępny bezpłatnie, Android zrzesza ogromną społeczność programistów piszących aplikacje, które poszerzają funkcjonalność tych urządzeń. Środowisko programistyczne Android pozwala programistom pisać oprogramowanie bez konieczności sięgania do niższych warstw systemu, dostarczając im bogaty interfejs użytkownika, włączając możliwość dostępu do Internetu, serwerów baz danych, drukarek i innych urządzeń peryferyjnych. 

Tematem pracy jest "Testowalność aplikacji mobilnych na platformę Android". Zauważono, że aplikacje napisane dla tego systemu nie są łatwe do testowania. Standardowa architektura, z~której korzysta obecnie większość programistów, nie pozwala na sprawne pisanie testów jednostkowych. Jeżeli test aplikacji rozpoczyna się od testów integracyjnych, nakład pracy będzie zdecydowanie większy, niż gdy zastosowany zostanie schemat standardowy, czyli zaczynając od testów jednostkowych, poprzez testy integracyjne, następnie systemowe, a~kończąc na akceptacyjnych. 

Pozbawiając się możliwości zastosowania testów jednostkowych na wczesnym etapie projektu z~powodu źle zaprojektowanej struktury aplikacji, projektanci ryzykują utratę jakości, a~co za tym idzie - utratę zaufania klientów do oprogramowania. Celem pracy jest przedstawienie alternatywnego podejścia do architektury aplikacji z~przeznaczeniem dla systemu Android, a~także przedstawienie alternatywnego sposobu podejścia do całego procesu pisania oprogramowania, które pomogłoby w~podniesieniu testowalności oprogramowania na ten system. Z~obserwacji autora wynika, że bardzo łatwo jest napisać zły i~nietestowalny kod dla Androida. Warto jednak wiedzieć, że zastosowanie pewnych procesów i~metodologii może znacznie poprawić testowalność aplikacji dla tego systemu. 

W części pierwszej pracy, zawierającej trzy rozdziały, przedstawione zostanie aktualne, szeroko stosowane podejście do programowania aplikacji dla systemu Android. Rozdział drugi zawiera ogólne informacje na temat Androida: historię powstania i~rozwoju systemu, analizę aktualnej popularności w~segmencie urządzeń przenośnych, a~także opis możliwości, jakie oferuje dla programistów. Rozdział trzeci dotyczy przede wszystkim testowania. Przypomniane zostaną podstawowe pojęcia testerskie, definicja testowalności i~pielęgnowalności oprogramowania, przeprowadzona analiza obszarów testowych i~rodzajów testów oraz wprowadzenie do zwinnych podejść w~procesie weryfikacji. W~rozdziale czwartym autor opisuje kluczowy problem dla tej publikacji, czyli problem testowalności aplikacji pisanych dla Androida. Przypomina pojęcia architektury systemowej i~softwarowej i~opisuje aktualnie stosowaną architekturę aplikacji Android. Następnie naświetlone zostają trudności, jakie napotyka się podczas testowania napisanego w~ten sposób oprogramowania.

Część druga to propozycja rozwiązania problemu nakreślonego w~rozdziale czwartym. W~rozdziale piątym autor proponuje zastosowanie techniki \textit{Test Driven Development} w~procesie tworzenia aplikacji oraz zachęca do wykorzystania jednej z~trzech opisanych koncepcji uporządkowania architektury systemowej: \textit{The Clean Architecture} autorstwa Roberta Cecila Martina. Proponuje również zwinne podejście w~przypadku modyfikacji już istniejących aplikacji. Rozdział szósty to analiza testowalności jednej z~gotowych aplikacji, napisanej raz w~architekturze standardowej, i~powtórnie z~wykorzystaniem podejścia \textit{TDD} oraz~wspomnianej \textit{Clean Architecture}.



