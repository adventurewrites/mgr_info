\chapter{Wstęp}
\label{wstep}

Pojęcie \textit{Android} ostatnio kojarzone jest przede wszystkim z rynkiem nowoczesnych urządzeń elektronicznych. Są to smartfony, tablety, netbooki, odbiorniki GPS, zegarki, a także telewizory, lodówki, pralki i inne urządzenia wykorzystywane obecnie w gospodarstwach domowych. Jako system operacyjny dostępny nieodpłatnie, Android zrzesza przy sobie ogromną społeczność developerów piszących aplikacje, które poszerzają funkcjonalność urządzeń.  Aplikacje Android pozwalają developerom wykonywać te czynności bez konieczności sięgania do niższych warstw systemu. W zamian framework Androida dostarcza developerom bogate środowisko użytkownika, które pozwala na dostęp do różnych udogodnień, jakie urządzenie z tym systemem jest w stanie programiście zaoferować. 

Tematem pracy jest "Testowalność aplikacji mobilnych na platformę Android". Z doświadczenia własnego autora oraz innych testerów można wnioskować, że aplikacje napisane dla tego systemu nie są łatwe do testowania. Standardowa architektura, z której korzysta obecnie większość programistów, nie pozwala na sprawne pisanie testów jednostkowych. Jeżeli test aplikacji rozpoczyna się od testów integracyjnych, nakład pracy będzie zdecydowanie większy, niż gdy zastosowany zostanie schemat standardowy, czyli zaczynając od \textit{Unit Tests}, kontynuując poprzez testy integracyjne, następnie systemowe, a kończąc na akceptacyjnych. 

Pozbawiając się możliwości zastosowania testów jednostkowych na wczesnym etapie projektu z powodu źle zaprojektowanej struktury aplikacji, ryzykujemy utratę jakości, a co za tym idzie - utratę zaufania klientów do naszego oprogramowania. Celem pracy jest przedstawienie alternatywnego podejścia do architektury aplikacji z przeznaczeniem dla systemu Android, a także przedstawienie alternatywnego sposobu podejścia do procesu pisania oprogramowania, które pomogłoby w podniesieniu testowalności oprogramowania na ten system. Z obserwacji autora wynika, że bardzo łatwo jest napisać zły i nietestowalny kod dla Androida. Warto jednak wiedzieć, że zastosowanie pewnych procesów i metodologii może znacznie poprawić testowalność naszych aplikacji. 

W części pierwszej pracy, zawierającej trzy rozdziały, przedstawione zostanie aktualne, szeroko stosowane podejście do programowania aplikacji dla systemu Android. Rozdział drugi zawiera ogólne informacje na temat Androida: historię powstania i rozwoju systemu, analizę aktualnej popularności w segmencie urządzeń przenośnych, a także opis możliwości, jakie oferuje dla programistów. Rozdział trzeci dotyczy przede wszystkim testowania. Przypomniane zostaną podstawowe pojęcia testerskie, definicja testowalności i pielęgnowalności oprogramowania, przeprowadzona analiza obszarów testowych i rodzajów testów oraz wprowadzenie do zwinnych podejść w procesie weryfikacji. W rozdziale czwartym autor opisuje kluczowy problem dla tej publikacji, czyli problem testowalności aplikacji pisanych dla Androida. Przypomina pojęcia architektury systemowej i softwarowej i opisuje aktualnie stosowaną architekturę aplikacji Android. Następnie naświetlone zostają trudności, jakie napotyka się podczas testowania napisanego w ten sposób oprogramowania.

Część druga to propozycja rozwiązania problemu nakreślonego w rozdziale czwartym. W rozdziale piątym autor proponuje zastosowanie techniki \textit{Test Driven Development} w procesie tworzenia aplikacji oraz zachęca do wykorzystania jednej z trzech opisanych koncepcji uporządkowania architektury systemowej: \textit{The Clean Architecture}. Proponuje również zwinne podejście w przypadku modyfikacji już istniejących aplikacji. Rozdział szósty to analiza testowalności jednej z gotowych aplikacji, napisanej raz w architekturze standardowej, i powtórnie z wykorzystaniem podejścia \textit{TDD} i wspomnianej \textit{Clean Architecture}.

Przy pisaniu pracy autor wykorzystywał zarówno pozycje książkowe, jak i artykuły blogowe takich znanych w świecie programistów osób, jak Mike Cohn, Alistair Cockburn, czy Robert Cecil Martin, czyli popularny \textit{Uncle Bob}. Wykorzystano również dokumentację systemową systemu Android, Sylabusa fundacji ISTQB oraz wiele informacji znalezionych na blogach programistów. Definicje niektórych pojęć wyjaśnione zostały z pomocą Wikipedii. 

Zdaniem autora, cel pracy został osiągnięty w około siedemdziesięciu procentach. Spowodowane to jest faktem, że w części doświadczalnej zajęto się tylko problemem testowalności w zakresie testów jednostkowych i wczesnych testów integracyjnych. Autor nie miał niestety możliwości przeanalizowania testowalności aplikacji Android na etapie testów systemowych czy akceptacyjnych. Natomiast można z dużym prawdopodobieństwem stwierdzić, że poprawienie testowalności w zakresie testów jednostkowych spowoduje jej polepszenie również na dalszych etapach procesu testowego.

Na koniec autor chciałby podziękować swoim promotorom za fachowe komentarze i nieocenioną pomoc przy rozwiązywaniu problemów technicznych i administracyjnych, bez której praca ta prawdopodobnie by nie powstała.


