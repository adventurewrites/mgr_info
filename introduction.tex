\chapter*{Wstęp}
\label{wstep}

Pojęcie \textit{android} kojarzone jest przede wszystkim z~rynkiem nowoczesnych urządzeń elektronicznych. Są to smartfony, tablety, netbooki, odbiorniki GPS, zegarki, a~także telewizory, lodówki, pralki i~inne urządzenia wykorzystywane obecnie. Android to również nazwa firmy, system operacyjny, projekty \textit{Open Source\footnote{Otwarte oprogramowanie (ang. open source movement, dosł. ruch otwartych źródeł) – odłam ruchu wolnego oprogramowania (ang. free software), który proponuje nazwę open source software jako alternatywną dla free software, głównie z~przyczyn praktycznych, a~nie filozoficznych. Obok darmowego udostępniania może ono być sprzedawane i~wykorzystywane w~sposób komercyjny. Sposób osiągania zysków można także ograniczyć do sprzedaży dodatkowych usług, takich jak szkolenia z~obsługi, wsparcie klienta czy dostęp do dodatkowych rozszerzeń, wtyczek, dodatków i~modułów. Możliwe jest też wykorzystanie bezpłatnej wersji open source, jako sposób na zachęcenie do kupna bardziej rozbudowanej wersji dostarczanej na licencji komercyjnej. \textit{"Otwarte oprogramowanie", Wikipedia, 2016}}}, a~nawet społeczność programistów. Jako system operacyjny dostępny bezpłatnie, Android zrzesza ogromną społeczność programistów piszących aplikacje, które poszerzają funkcjonalność tych urządzeń. Dostarcza programistom bogate środowisko użytkownika pozwalając pisać oprogramowanie bez konieczności sięgania do niższych warstw systemu, włączając możliwość dostępu do Internetu, serwerów baz danych, drukarek i~innych urządzeń peryferyjnych. 

Zauważono jednak, że~aplikacje napisane dla tego systemu nie są łatwe do testowania. Standardowa architektura, z~której korzysta obecnie większość programistów, nie pozwala na sprawne pisanie testów jednostkowych \cite{tematpracy}. Jeżeli proces testowania aplikacji rozpoczyna się od testów integracyjnych, nakład pracy będzie zdecydowanie większy, niż gdy zastosowany zostanie schemat standardowy, czyli zaczynając od testów jednostkowych, poprzez testy integracyjne, następnie systemowe, a~kończąc na akceptacyjnych. Według autora do tej pory nie podjęto wystarczającej ilości prób rozwinięcia tego problemu. Stąd tematem pracy jest "Testowalność aplikacji mobilnych na platformę Android". 

Pozbawiając się możliwości zastosowania testów jednostkowych na wczesnym etapie projektu z~powodu źle przemyślanej struktury aplikacji, projektanci ryzykują utratę jakości, a~co za tym idzie - utratę zaufania klientów do oprogramowania. Celem pracy jest przebadanie różnych podejść do architektury aplikacji z~przeznaczeniem dla systemu Android i~sprawdzenie jak one wpływają na testowalność tego systemu. Z~obserwacji autora wynika, że~bardzo łatwo jest napisać zły i~nietestowalny kod dla Androida. Warto jednak wiedzieć, że~zastosowanie pewnych procesów i~metodologii może znacznie poprawić testowalność aplikacji dla tego systemu. 

W części pierwszej pracy przedstawiono aktualne, szeroko stosowane podejście do programowania aplikacji dla systemu Android. Rozdział pierwszy zawiera ogólne informacje na temat Androida: historię powstania i~rozwoju systemu, analizę aktualnej popularności w~segmencie urządzeń przenośnych, a~także opis możliwości, jakie oferuje dla programistów. Rozdział drugi dotyczy przede wszystkim testowania. Przypomniane zostaną podstawowe pojęcia testerskie, definicja testowalności i~pielęgnowalności oprogramowania, przeprowadzona analiza obszarów testowych i~rodzajów testów oraz wprowadzenie do zwinnych podejść w~procesie weryfikacji. W~rozdziale trzecim autor opisuje kluczowy problem dla tej publikacji, czyli problem testowalności aplikacji pisanych dla Androida. Przypomina pojęcia architektury systemowej i~programowej oraz~opisuje aktualnie stosowane podejście przy projektowaniu aplikacji Android. Następnie naświetlone zostają trudności, jakie napotyka się podczas testowania napisanego w~ten sposób oprogramowania.

Część druga to propozycja rozwiązania problemu nakreślonego w~rozdziale trzecim. W~rozdziale czwartym autor proponuje zastosowanie techniki \textit{Test Driven Development} w~procesie tworzenia aplikacji oraz zachęca do wykorzystania jednej z~trzech opisanych koncepcji uporządkowania architektury systemowej: \textit{The Clean Architecture} autorstwa Roberta Cecila Martina. Proponuje również zwinne podejście w~przypadku modyfikacji już istniejących aplikacji. Rozdział piąty to analiza testowalności jednej z~gotowych aplikacji, napisanej raz w~architekturze standardowej, i~powtórnie z~wykorzystaniem podejścia \textit{TDD} oraz~wspomnianej \textit{The Clean Architecture}.



