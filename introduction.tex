\chapter{Wprowadzenie}
\label{wstep}

Pojęcie \textit{android} kojarzone jest przede wszystkim z~rynkiem nowoczesnych urządzeń elektronicznych. Są to smartfony, tablety, netbooki, odbiorniki GPS, zegarki, a~także telewizory, lodówki, pralki i~inne urządzenia wykorzystywane codziennie przez ludzi na całym świecie. Android to również nazwa firmy, system operacyjny oraz projekty \textit{Open Source\footnote{Otwarte oprogramowanie (ang. open source movement, dosł. ruch otwartych źródeł) – odłam ruchu wolnego oprogramowania (ang. free software), który proponuje nazwę open source software jako alternatywną dla free software, głównie z~przyczyn praktycznych, a~nie filozoficznych. Obok darmowego udostępniania może ono być sprzedawane i~wykorzystywane w~sposób komercyjny \cite{website:wikipedia}.}}. Jako system operacyjny dostępny bezpłatnie, Android zrzesza ogromną społeczność programistów piszących aplikacje, które poszerzają funkcjonalność tych urządzeń. Dostarcza programistom bogate środowisko użytkownika pozwalając pisać oprogramowanie bez konieczności sięgania do niższych warstw systemu, włączając możliwość dostępu do Internetu, serwerów baz danych, drukarek i~innych urządzeń peryferyjnych. W~pierwszym kwartale 2016 roku w~internetowym sklepie Google Play (wcześniej Android Market)\footnote{Google Play (dawniej Android Market) – internetowy sklep Google z~aplikacjami, grami, muzyką, książkami, magazynami, filmami i~programami TV. Treści ze sklepu są przeznaczone do korzystania za pomocą urządzeń działających pod kontrolą systemu operacyjnego Android, ale z~niektórych można także korzystać na laptopie czy komputerze stacjonarnym \cite{website:wikipedia}.} dostępnych było ponad 1,9 miliona aplikacji. 

Mimo tak dużej popularności platformy Android oraz licznych publikacji dotyczących programowania na nią, okazuje się, że temat automatycznego testowania opracowywanych aplikacji jest często pomijamy w literaturze. Okazuje się również, że stosowane powszechnie podejście do tworzenia nowych aplikacji nie pozwala na sprawne pisanie testów jednostkowych \cite{tematpracy}. W konsekwencji programiści często korzystają z testów integracyjnych, gdzie nakład pracy jest dużo większy. Dodatkowo, pozbawiając się możliwości zastosowania testów jednostkowych na wczesnym etapie projektu z~powodu źle przemyślanej struktury aplikacji, projektanci ryzykują utratę jakości, a~co za tym idzie - utratę zaufania klientów do oprogramowania. Wychodząc tym problemom naprzeciw, autor w tej pracy próbuje dowieść, że zastosowanie odpowiedniej architektury oraz odpowiedniego podejścia do pisania oprogramowania może ułatwić testowanie aplikacji napisanych dla Androida. Celem jest przebadanie różnych podejść do architektury aplikacji z~przeznaczeniem dla systemu Android i~sprawdzenie jak one wpływają na testowalność tego systemu. 

Praca została podzielona na dwie części. W części pierwszej autor prezentuje aktualne, szeroko stosowane podejście do programowania aplikacji dla systemu Android. Rozdział drugi zawiera ogólne informacje na temat Androida: historię powstania i~rozwoju systemu, analizę aktualnej popularności w~segmencie urządzeń przenośnych, a~także opis możliwości, jakie oferuje dla programistów. Rozdział trzeci dotyczy przede wszystkim testowania. Przypomniane zostaną podstawowe pojęcia testerskie, definicja testowalności i~pielęgnowalności oprogramowania, przeprowadzona analiza obszarów testowych i~rodzajów testów oraz wprowadzenie do zwinnych podejść w~procesie weryfikacji. W~rozdziale czwartym autor opisuje problem testowalności aplikacji pisanych dla Androida. Przypomina również pojęcia architektury systemowej i~programowej oraz~opisuje aktualnie stosowane podejście przy projektowaniu aplikacji Android. Następnie naświetlone zostają trudności, jakie napotyka się podczas testowania napisanego w~ten sposób oprogramowania.

W części drugiej, praktycznej, zostaje przedstawiona propozycja rozwiązania problemu nakreślonego w~rozdziale czwartym. W~rozdziale piątym autor proponuje zastosowanie techniki \textit{Test Driven Development} w~procesie tworzenia aplikacji oraz zachęca do wykorzystania jednej z~trzech opisanych koncepcji uporządkowania architektury systemowej: \textit{The Clean Architecture} autorstwa Roberta Cecila Martina. Proponuje również zwinne podejście w~przypadku modyfikacji już istniejących aplikacji. Rozdział szósty to analiza testowalności jednej z~gotowych aplikacji, przy pisaniu której wykorzystano wyżej wymienione techniki.

