\chapter*{Zakończenie}
\label{zakonczenie}

Poniższa praca zawiera analizę testowalności aplikacji na platformę Android. Praca została podzielona na dwie części. W~pierwszej autor wprowadza i~wyjaśnia podstawowe pojęcia związane z~programowaniem na platformę Android oraz testowalnością oprogramowania. Okazuje się, że powszechnie stosowane podejście projektowe przy realizacji aplikacji mobilnych na platformę Android znacznie utrudnia wydajne pisanie testów jednostkowych, a~cały ciężar testowania przeniesiony jest na testy integracyjne, systemowe, a~nawet akceptacyjne. W~części drugiej autor szczegółowo opisuje proces projektowy, który znacząco ułatwia utrzymanie i~opracowywanie aplikacji Andorid \cite{website:cecil:blog}. Następnie porównuje wybrane rozwiązania architektoniczne w~kontekście automatycznego testowania. Otrzymane wyniki są bardzo obiecujące i~pozwalają na wyciągnięcie wielu ciekawych wniosków.

Testy automatyczne powinny przyczyniać się do poprawy jakości dostarczając szybkiej informacji zwrotnej na temat działania programu. Z~przeprowadzonej analizy można wnioskować, że~największą korzyść przynosi automatyzacja testów jednostkowych. Są relatywnie łatwe do napisania, wykonują się w~ułamkach sekund i~są również łatwe do modyfikacji. Okazuje się, że~zastosowanie architektury \textit{The Clean Architecture} oraz techniki \textit{Test Driven Development} w~przypadku badanej aplikacji pozwala na napisanie 6 razy mniej testów jednostkowych niż w~przypadku tej samej aplikacji napisanej w~sposób standardowy, co zostało zaprezentowane na wykresie \ref{fig:app_ut_liczba}. Przekłada się to zarówno na czas wykonywania tych testów (wykres \ref{fig:app_ut_czas}), co ma niebagatelne znaczenie przy automatyzacji zestawów testowych i~wyborze zakresu dla testów regresywnych, jak i~ich pielęgnowalność. 

W przypadku testów integracyjnych analiza wykazała, że zysk może być jeszcze większy. W~granicznym przypadku czas ich wykonania, jak pokazuje wykres \ref{fig:app_int_czas}, może zostać skrócony nawet czterdziestokrotnie.

Cel pracy został osiągnięty. Wykonane eksperymenty potwierdzają wszystkie postawione przez autora tezy. W~przyszłości planuje się rozszerzenie badań na testy systemowe i~akceptacyjne. Teraz nie było to możliwe, ponieważ w~badanej aplikacji nie są one zdefiniowane, a~autor nie miał dostępu zarówno do wymagań systemowych, jak i~do wymogów na poziomie użytkownika. 


