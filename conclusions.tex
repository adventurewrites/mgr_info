\chapter*{Zakończenie}
\label{zakonczenie}

Temat tej pracy to "Testowalność aplikacji mobilnych na platformę Android". Jak już wspomniano na początku tej publikacji, aplikacje napisane dla tego systemu są niezwykle trudne do testowania. Standardowa architektura, stosowana powszechnie, nie pozwala na sprawne pisanie testów jednostkowych, a~jeżeli zostaną one już zaprojektowane, to ich pielęgnacja jest utrudniona ze względu na duży stopień zależności między poszczególnymi modułami programu. Zauważono również, że~jeżeli test aplikacji rozpoczyna się od testów integracyjnych, nakład pracy na przetestowanie tego samego obszaru funkcjonalnego może być zdecydowanie większy, niż gdy zastosowany zostanie schemat standardowy. Znacznie łatwiej testować aplikację według procedury: najpierw testy jednostkowe, potem integracyjne, systemowe i~na sam koniec testy akceptacyjne. 

Pozbawiając się możliwości zastosowania testów jednostkowych na wczesnym etapie projektu z~powodu źle zaprojektowanej struktury aplikacji, jej projektanci ryzykują utratę jakości, a~co za tym idzie - utratę zaufania klientów do oprogramowania. Celem pracy było przeanalizowanie alternatywnych podejść do projektowania architektury aplikacji z~przeznaczeniem dla systemu Android, a~także przeanalizowanie alternatywnego sposobu podejścia do całego procesu tworzenia oprogramowania, które pomogłoby w~podniesieniu testowalności programów dla tego systemu. W~rozdziale \ref{propozycja_rozwiazania} przedstawiono tezę, że zastosowanie pewnych procesów i~metodologii może znacznie poprawić testowalność aplikacji androidowych.

W części doświadczalnej - rozdziale \ref{analiza_testow} - przeanalizowano przykładową aplikację dla Systemu Android zaprojektowaną na dwa sposoby. Wersja pierwsza to aplikacja napisana w~standardowej architekturze, wersja druga to ten sam program napisany przy wykorzystaniu \textit{The Clean Architecture}. W tym celu zdecydowano się wykorzystać
aplikację \textit{JSON Web Token Authentication for Android} napisaną przez Victora Albertosa \cite{website:victor:aplication} , a~której źródła udostępnione są w~serwisie GitHub na licencji \textit{open source}.

Cel pracy nie został osiągnięty w~stu procentach. W~części doświadczalnej badania ograniczono tylko do analizy testowalności w~zakresie testów jednostkowych i~wczesnych testów integracyjnych pomijając przegląd na etapie testów systemowych i~akceptacyjnych. Spowodowane to było brakiem dostępu do wymagań systemowych i~wymagań klienta dla badanej aplikacji. Również wybór dostępnych do badań programów był ograniczony: trudno było znaleźć taki program, którego autorzy udostępniliby jego kod źródłowy w~dwóch architekturach: standardowej oraz \textit{The Clean Architecture}. Jednakże już na tej podstawie można ocenić, że zastosowanie \textit{Test Driven Development} i~architektury \textit{Clean Architecture} poprawia testowalność i~powoduje ułatwienie dalszego procesu testowego w~przypadku aplikacji dla systemu Android. Zdaniem autora, warto byłoby jednak problem testów integracyjnych, systemowych i~akceptacyjnych zbadać osobno.

O ile dostępnych jest wiele publikacji na temat \textit{Test Driven Development}, o~tyle dostęp do dokumentacji alternatywnych architektur systemowych jest mało. Większość z nich, to artykuły blogowe artykułów propozycji. Autor zdecydował się wybrać podejście \textit{The Clean Architecture} Roberta Cecila Martina dlatego, że oprócz informacji zawartych na jego blogu, mógł skorzystać również z~książek pisanych przez niego bardzo przyjaznym dla programistów językiem. Większość publikacji, z~których korzystano w~trakcie tworzenia tej pracy, to źródła obcojęzyczne. Zachęca to tym bardziej do popularyzacji tego zagadnienia w~Polsce.

